documentclass[12pt, a4paper]{article}
usepackage{graphicx}
usepackage{amsmath}
usepackage{float}
usepackage{minted}

title{EE2703 ASSIGNMENT 4}
author{Nandha Varman EE19B043}
date{}

DeclareUnicodeCharacter{2212}{-}
begin{document}

maketitle

section{Outline}
In this assignment, we will use the concept of Fourier series to approximate functions $exp(x)$ and $cos(cos(x))$.

We will employ two methods to find the coefficients, namely textbf{Direct Integration} and textbf{Least Squares Estimation}. We will also compare the results of the two methods and then see how close the Fourier approximation is to the actual functions.


section{Theory}
textit{Any periodic function can be written as a sum of shifted sinusoids}
We will apply this idea of Fourier series in this assignment. Namely a function can be approximated in the domain $[0,2pi]$ by repeating its values in $[0,2pi]$, to make it a periodic function with periodic $2pi$.

The Fourier series of a function $f(x)$ is given by

begin{equation}
    f(x) = a_0 + sum_{k=1}^{infty}a_kcos(kx) + b_ksin(kx)
end{equation}
where,
begin{equation}
        a_0 = frac{1}{2pi}int_{0}^{2pi}f(x)dx
end{equation}
begin{equation}
    begin{split}
    a_k = frac{1}{pi}int_{0}^{2pi}f(x)cos(kx)dx
    b_k = frac{1}{pi}int_{0}^{2pi}f(x)sin(kx)dx
    end{split}
end{equation}

section{Tasks}
    subsection{Part 1}
    begin{itemize}
        item Define Python functions for the two functions $e^x$ and $cos(cos(x))$ which return a vector (or scalar) value.
        item Plot the functions over the interval $[-2pi,4pi)$.
        item Discuss periodicity of both functions.
        item Plot the expected functions from Fourier series.
    end{itemize}
    
subsubsection{Code}
begin{minted}
[
frame=lines,
framesep=2mm,
baselinestretch=1.2,
bgcolor=white,
fontsize=footnotesize,
]
{python}
#importing necessary modules
import matplotlib.pyplot as plt
import numpy as np
from scipy.integrate import quad

PI = np.pi

# functions for exp(x) and cos(cos(x))
def f1(x)
    return np.exp(x)
def f2(x)
    return np.cos(np.cos(x))

t = np.linspace(-2PI, 4PI, 500 )
t = t[0-1]

# lists for locations and labels for ticks
locs = [-2PI, -32PI, -PI,-PI2, 0, PI2, PI, 3PI2, 2PI, 5PI2, 3PI, 7PI2, 4PI]
labels = ['$-2pi$', '$-3pi2$', '$-pi$','$-pi2$', 0, '$pi2$', '$pi$', '$3pi2$', '$2pi$', '$5pi2$', '$3pi$','$7pi2$', '$4pi$']

# plotting exp(x) as a semilog plot
plt.figure(1)
plt.xticks(locs, labels) 
plt.semilogy(t,f1(t),'k', label = 'True function')
plt.semilogy(t,f1(t%(2PI)),'r', label = 'Expected')
plt.legend(loc='upper right')
plt.title('$exp(x) semilog plot$')
plt.grid()
plt.show()

# plotting cos(cos(x))
plt.figure(2)
plt.xticks(locs, labels) 
plt.plot(t,f2(t),'k', label = 'True function')
plt.plot(t,f2(t%(2PI)),'--', label = 'Expected')
plt.legend(loc='upper right')
plt.title('$cos(cos(x))$')
plt.grid()
plt.show()

end{minted}

subsubsection{Plots}
begin{figure}[H]
    centering
    begin{tabular}{cc}
        includegraphics[scale=0.5]{1.png} &                includegraphics[scale=0.5]{2.png}
    end{tabular}
end{figure}

subsubsection{Observations}
begin{enumerate}
    item From the above plots, you can easily see that $cos(cos(x))$ is periodic with period $2pi$,and $e^x$ isn't periodic and rises monotonically.
    item From the Fourier series, we expect that the Fourier approximation of the function would be repetitions of the function between $[0, 2pi)$. 
end{enumerate}

subsection{Part 2}
begin{itemize}
    item Obtain the first 51 coefficients, i.e., $a_0$, $a_1$, $b_1$,ldots for $e^x$ and $cos(cos(x))$ using the texttt{quad} function.
    item Calculate the function using those coefficients and compare with original functions graphically.
end{itemize}
 
subsubsection{Code}
begin{minted}
[
frame=lines,
framesep=2mm,
baselinestretch=1.2,
bgcolor=white,
fontsize=footnotesize,
]
{python}

def cos_coeff(x, k, h)
    return h(x)np.cos(kx)
def sin_coeff(x, k, h)
    return h(x)np.sin(kx)

# function to find the first 51 fourier series coefficients
def fourier_coeff_51(h)
    coeff = np.zeros(51)
    coeff[0] = quad(cos_coeff, 0, 2PI, args =(0, h))[0](2PI)
    for i in range(1,26)
        coeff[2i-1] = quad(cos_coeff, 0, 2PI, args =(i, h))[0](PI)
        coeff[2i] = quad(sin_coeff, 0, 2PI, args =(i, h))[0](PI)
    return  coeff

# arrays holding the first 51 FS coefficients in the form [ a0, a1, b1, a2, b2,..]
exp_coeff = fourier_coeff_51(f1)
coscos_coeff = fourier_coeff_51(f2)

end{minted}
subsubsection{Observations}
The function mintinline{python}{fourier_coeff_51(h)} takes in the name of a function as argument and returns the first 51 coefficients of the function texttt{h}, in order, as an array.

subsection{Part 3}
Make two  different  plots  for  each  function  using  texttt{semilogy}  and  texttt{loglog} and plot the magnitude of the coefficients versus n.

subsubsection{Code}

begin{minted}
[
frame=lines,
framesep=2mm,
baselinestretch=1.2,
bgcolor=white,
fontsize=footnotesize,
]
{python}
# generating lists for location and labels for the coefficents plot
n_locs = range(51)
n_labels = ['$a_0$']
for i in range(1,26)
    n_labels.append('$a_{'+str(i)+'}$')
    n_labels.append('$b_{'+str(i)+'}$') 

# semi-log plot for FS coefficients of exp(x)
plt.figure(3)
plt.xticks(n_locs,n_labels , rotation = '90')
plt.tick_params(axis='x', labelsize=7)
plt.semilogy(0, np.abs(exp_coeff[0]), 'ro')
plt.semilogy(np.arange(1,51,2), np.abs(exp_coeff[12]),'ro', label = 'Cosine coefficients')
plt.semilogy(np.arange(2,51,2), np.abs(exp_coeff[22]),'bo', label = 'Sine coefficents')
plt.legend(loc='upper right')
plt.title('$exp(x)$ fourier coefficientssemilog plot')
plt.grid()
plt.show()

# log-log plot for FS coefficients of exp(x)
plt.figure(4)
plt.loglog(0, np.abs(exp_coeff[0]), 'ro')
plt.loglog(np.arange(1,51,2), np.abs(exp_coeff[12]),'ro', label = 'Cosine coefficients')
plt.loglog(np.arange(2,51,2), np.abs(exp_coeff[22]),'bo', label = 'Sine coefficents')
plt.legend(loc='upper right')
plt.title('$exp(x)$ fourier coefficientslog-log plot')
plt.grid()
plt.show()

# semi-log plot for FS coefficients of cos(cos(x))
plt.figure(5)
plt.xticks(n_locs,n_labels , rotation = '90')
plt.tick_params(axis='x', labelsize=7)
plt.semilogy(0, np.abs(coscos_coeff[0]), 'ro')
plt.semilogy(np.arange(1,51,2), np.abs(coscos_coeff[12]),'ro', label = 'Cosine coefficients')
plt.semilogy(np.arange(2,51,2), np.abs(coscos_coeff[22]),'bo', label = 'Sine coefficents')
plt.legend(loc='upper right')
plt.title('$cos(cos(x))$ fourier coefficientssemilog plot')
plt.grid()
plt.show()

# log-log plot for FS coefficients of cos(cos(x))
plt.figure(6)
plt.loglog(0, np.abs(coscos_coeff[0]), 'ro')
plt.loglog(np.arange(1,51,2), np.abs(coscos_coeff[12]),'ro', label = 'Cosine coefficients')
plt.loglog(np.arange(2,51,2), np.abs(coscos_coeff[22]),'bo', label = 'Sine coefficents')
plt.legend(loc='upper right')
plt.title('$cos(cos(x))$ fourier coefficientslog-log plot')
plt.grid()
plt.show()
end{minted}

subsubsection{Plots}
begin{figure}[H]
    centering
    begin{tabular}{cc}
        includegraphics[scale=0.5]{3.png} &                includegraphics[scale=0.5]{4.png}
        includegraphics[scale=0.5]{5.png} &                includegraphics[scale=0.5]{6.png}
    end{tabular}
end{figure}

subsubsection{Observations}
begin{enumerate}
    item From the above plots we observe that almost all $b_n$ coefficients are very close to 0 for $cos(cos(x))$. This is expected due to the odd nature of the integrand between $[-pi,pi)$.

    It does not become exactly zero due to approximations in the implementation of the texttt{quad} function.

    item Rate of decay of Fourier coefficients is determined by how smooth the  function is. If a function is infinitely differentiable then its Fourier coefficients decay very fast. But if the $k^{th}$ derivative of the function $f$, denoted by $f^{(k)}$, is discontinuous, then the rate of decay of the Fourier coefficients is only $frac{1}{n^k}$.

    $cos(cos(x))$ is infinitely differentiable, hence it's Fourier coefficients decay very fast, while that of $e^x$ decay very slowly due to the discontinuity in the Fourier approximation of the function at $2npi$.
end{enumerate}

subsection{Part 4}
begin{itemize}
    item Use textit{Least Squares estimation} to find the Fourier coefficients of the functions, using texttt{scipy.linalg.lstsq} .          
    item Build the coefficient matrix $A$ and the constant matrix $b$.
end{itemize}

subsubsection{Code}
begin{minted}
[
frame=lines,
framesep=2mm,
baselinestretch=1.2,
bgcolor=white,
fontsize=footnotesize,
]
{python}

# function to find the first 51 coefficients by least squares estimation
def Lstsq_FS_coeff(f)
    x=np.linspace(0,2PI,401)
    x=x[-1] # drop last term to have a proper periodic integral
    b=f(x) # f has been written to take a vector
    A=np.zeros((400,51)) # allocate space for A
    A[,0]=1 # col 1 is all ones
    for k in range(1,26)
        A[,2k-1]=np.cos(kx) # cos(kx) column
        A[,2k]=np.sin(kx) # sin(kx) column
    #endfor
    ls_coeffs=np.linalg.lstsq(A,b,rcond = -1)[0] # the ’[0]’ is to pull out the
    # best fit vector. lstsq returns a list.
    return ls_coeffs

# arrays holding the first 51 FS coefficients obtained by least squares estimation
exp_coeff_lstsq = Lstsq_FS_coeff(f1)
coscos_coeff_lstsq = Lstsq_FS_coeff(f2)



end{minted}



subsection{Part 5}
begin{itemize}
    item Compare the coefficients obtained through the textit{least squares method} and the textit{direct integration} method.
    item Find the maximum deviation between the coefficients obtained in the two methods.
end{itemize}

subsubsection{Code}

begin{minted}
[
frame=lines,
framesep=2mm,
baselinestretch=1.2,
bgcolor=white,
fontsize=footnotesize,
]
{python}
# semi-log comparison plot of the FS coefficents obtained by integrationa and
# Least squares estimation for exp(x)
plt.figure(7)
plt.xticks(n_locs, n_labels, rotation = '90')
plt.tick_params(axis='x', labelsize=7)
plt.semilogy(range(51), np.abs(exp_coeff),'ro', label = 'by Integration')
plt.semilogy(range(51), np.abs(exp_coeff_lstsq),'go', label = 'by Least Squares')
plt.legend(loc='upper right')
plt.title('$exp(x)$ fourier coefficients comparisonsemilog plot')
plt.grid()
plt.show()

# semi-log comparison plot of the FS coefficents obtained by integrationa and
# Least squares estimation for exp(x)
plt.figure(8)
plt.xticks(n_locs, n_labels, rotation = '90')
plt.tick_params(axis='x', labelsize=7)
plt.loglog(range(51), np.abs(exp_coeff),'ro', label = 'by Integration')
plt.loglog(range(51), np.abs(exp_coeff_lstsq),'go', label = 'by Least Squares')
plt.legend(loc='upper right')
plt.title('$exp(x)$ fourier coefficients comparisonlog-log plot')
plt.grid()
plt.show()

# semi-log comparison plot of the FS coefficents obtained by integrationa and
# Least squares estimation for cos(cos(x))
plt.figure(9)
plt.xticks(n_locs, n_labels, rotation = '90')
plt.tick_params(axis='x', labelsize=7)
plt.semilogy(range(51), np.abs(coscos_coeff),'ro', label = 'by Integration')
plt.semilogy(range(51), np.abs(coscos_coeff_lstsq),'go', label = 'by Least Squares')
plt.legend(loc='upper right')
plt.title('$cos(cos(x))$ fourier coefficients comparisonsemilog plot')
plt.grid()
plt.show()

# log-log comparison plot of the FS coefficents obtained by integrationa and
# Least squares estimation for cos(cos(x))
plt.figure(10)
plt.xticks(n_locs, n_labels, rotation = '90')
plt.tick_params(axis='x', labelsize=7)
plt.loglog(range(51), np.abs(coscos_coeff),'ro', label = 'by Integration')
plt.loglog(range(51), np.abs(coscos_coeff_lstsq),'go', label = 'by Least Squares')
plt.legend(loc='upper right')
plt.title('$cos(cos(x))$ fourier coefficients comparison loglog plot')
plt.grid()
plt.show()

# finding the maximum absolute deviation between the FS coefficients obtained by the two methods
abs_error_exp = [abs(exp_coeff[i]-exp_coeff_lstsq[i]) for i in range(len(exp_coeff))]
abs_error_coscos = [abs(coscos_coeff[i]-coscos_coeff_lstsq[i]) for i in range(len(coscos_coeff))]


print('Largest deviation for exp(x) '+str(max(abs_error_exp)))
print('Largest deviation for cos(cos(x)) '+str(max(abs_error_coscos)))
end{minted}

subsubsection{Plots}
begin{figure}[H]
    centering
    begin{tabular}{cc}
        includegraphics[scale=0.5]{7.png} &                includegraphics[scale=0.5]{8.png}
        includegraphics[scale=0.5]{9.png} &                includegraphics[scale=0.5]{10.png}
    end{tabular}
end{figure}

subsection{Part 6}
begin{itemize}
    item Computing $Acdot c$ from the estimated values of $c$ by textit{Least Squares Method} and plotting them.
end{itemize}
subsubsection{Code}

begin{minted}
[
frame=lines,
framesep=2mm,
baselinestretch=1.2,
bgcolor=white,
fontsize=footnotesize,
]
{python}
x = np.linspace(-2PI, 4PI, 1001)
x = x[-1]

A = np.zeros((1000,51))
A[,0] = 1
for k in range(1,26)
    A[,2k-1]=np.cos(kx)
    A[,2k]=np.sin(kx)

# plot showing reconstruction of exp(x) using the obtained FS coefficients 
plt.figure(11)
plt.xticks(locs, labels) 
plt.plot(x, np.matmul(A, exp_coeff ), 'go', label='Integration')
plt.plot(x, np.matmul(A, exp_coeff_lstsq), 'r', label='Least squares')
plt.title('Reconstruction of $exp(x)$')
plt.legend()
plt.grid()

# plot showing reconstruction of cos(cos(x)) using the obtained FS coefficients
plt.figure(12)
plt.xticks(locs, labels) 
plt.plot(x, np.matmul(A, coscos_coeff ), 'go', label='Integration')
plt.plot(x, np.matmul(A, coscos_coeff_lstsq), 'r', label='Least squares')
plt.title('Reconstruction of $cos(cos(x))$')
plt.legend()
plt.grid()

end{minted}
subsubsection{Plots}

begin{figure}[H]
    centering
    begin{tabular}{cc}
        includegraphics[scale=0.5]{11.png} &                includegraphics[scale=0.5]{12.png}
    end{tabular}
end{figure}

subsubsection{Observations}
begin{itemize}

    item
    We observe that there is a significant deviation for (e^{x}) at its discontinuities at (2npi). There are oscillations around the discontinuity points and their ripple amplitude decrease as we go close to discontinuity.This is called textbf{Gibbs phenomenon}.
    item
    Due to this, the original function and one which is reconstructed using
    least squares will not fit exactly.
    item
    And as we know that Fourier series is used to define periodic signals in frequency domain. (e^{x}) is a aperiodic signal and we can't define an aperiodic signal on an interval of finite length.
    item
    For (cos(cos(x))) the curves fit almost perfectly because the function itself is a periodic function and it is continuous everywhere, so we get very negligible deviation and can reconstruct the signal with just the Fourier coefficients.
end{itemize}
section{Conclusions}
The Fourier estimation of (e^x) does not match accurately with the function close to (0), but matches almost exactly in the case of (cos(cos(x))). This is due to the presence of a discontinuity at (x=0) for the periodic extension of (e^x). This discontinuity leads to non-uniform convergence of the
Fourier series, with different rates for both the functions.

The mismatch in the Fourier approximation of (e^x) is explained by Gibbs Phenomenon.

We can hence conclude that the Fourier Series Approximation Method works extremely well for smooth periodic functions, but gives bad results for discontinuous periodic functions.

end{document}